\section{Реализация}
\subsection{Инициализация}
В основном файле~--- \textbf{MechanicProject.cpp} реализован ввод данных. Пользователь вводит набор координат точек в формате (x,y) и заполняет матрицу весов, создаются необходимые объекты классов:
\begin{itemize}
    \item Объект класса \textit{Scene} (грубо говоря, уровень земли в окне \texttt{OpenGl})
    \item Объект класса \textit{RigidBody}, который хранит в себе данные о нашей фигуре - количество точек, их координаты, матрицу связей (весов)
    \item Объект класса \textit{Physical},  который отвечает за пересчет положения точек тела
\end{itemize}
Для хранения данных об отдельной точке был создан класс \textit{Point}. Также в данном классе реализована необходимая арфметика (перегружены операторы *, +, - и =, а также реализована функция поворота одной точки вокруг другой \texttt{Point::Rotate})
Особым, в смысле инициализации, является объект класса \textit{ModelSimulator}. Данный класс отвечает за активацию отрисовки объектов на сцене и на его метод \textit{render} должна ссылаться функция отрисовки \textit{Draw} OpenGl. Но \textit{Draw} не может принимать что-либо в качестве аргумента, поэтому создается глобальный указатель на объект этого класса.
\\После инициализаци данных производится инициализация окна \textbf{OpenGl}
\subsection{Иерархия классов и описание методов}
Для отрисовки я создал абстрактный базовый класс \textit{SceneObject}, от которого отнаслдовались два класса
\begin{itemize}
    \item \textit{Scene}
    \item \textit{RigidBody}
\end{itemize}
\textit{SceneObject} содержит только виртуальную функцию \textit{render}.
Класс \textit{Scene} отвечает только за отрисовку "уровня земли" в окне. В перспективе можно добавить другие объекты, с которыми будет взаимодействовать наша фигура.\\\\
Класс \textit{RigidBody}, как было уже сказано ранее, содержит в себе данные о фигуре. А также составляет массивы весов и длин ребер (по координатам точек и матрице связей), отрисывавает наше тело по положению точек.\\\\
Особый интерес представляют классы \textit{ModelSimulator} и \textit{Physical}. Класс \textit{ModelSimulator} отвечает за  обновление среды. То есть вызывает пересчет координат в методе \texttt{ModelSimulator::update}, а после вызывает функцию отрисовки тела \texttt{RigidBody::render} и сцены \texttt{Scene::render}.\\\\
Класс \textit{Physical} отвечает за математическую модель нашего физического процесса (падения фигуры), а если проще, то за пересчет координат точек нашей фигуры. На каждом шаге после проведения всех необходимых вычислений указанных в \textbf{разделе 3} производятся повороты и смещения точек относительно друг друга с помощью перегруженных операторов арифметических операций класса \textit{Point}, а также его метода \texttt{RigidBody::rotate}.


