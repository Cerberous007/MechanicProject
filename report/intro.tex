\section{Введение}
\subsection{Моделирование}
    Моделирование является важнейшим и неотъемлемым этапом процедуры проектирования современных механических устройств и систем.
    Термин "моделирование"~ весьма многогранен. Применнительно к техническим(в т.ч. механическим) системам под \textbf{моделированием}, будем понимать процесс, состоящий в выявлении основных свойств исследуемого объекта, построении моделей и их применении для прогнозирования поведения объекта. Математические модели в механики, как правило, представляются в виде систем дифференциальных уравнений

\subsection{Методы решения задачи моделирования}
	Для решения задачи математического моделирования в общем случае могут использоваться следующие методы:
	\begin{itemize}
		\item Аналитический методы;
		\item Численные методы;
	\end{itemize}

	\textbf{Аналитический метод}. Для решения аналитическим методом необходимо явное интегрирование дифференциального уравнения, что на практике может быть весьма сложной или вовсе невыполнеимой задачей.
	
	\textbf{Численные методы}. Существует множество алгоритмов численного решения дифференциальных уравнений, которые позволяют избавиться от проблемы сложного интегрирования и решают данную задачу с необходимой точностью.
	
	В проекте я буду использовать численные методы, чтобы избежать сложного интегрирования.