\section{Анализ}
\subsection{Тестирование}
Рассматривались различные начальные положения системы (смещение влево, смещение вправо, вниз и т.д). Падение моделируется достаточно приближенно к реальности. Так как изначально не было никаких внешних сил, связи предполагались идеальные. При таком подходе выбранный метод приближенного решения системы приводил к слишком большому накапливанию ошибок. То есть, если программа была запущена в течение длительного времени, накапливалась большая погрешность при пересчете координат и модель начинала выглядеть не естественной. 
\subsection{Анализ результатов}
Вышеуказанные неточности я решил с помощбю добавления внешних сил~--- сил трения. В таком случае фигура, в конечном итоге, принимала положение покоя(равновесия).
\subsection{Вывод}
Данная модель является очень гибкой. Ее можно использовать для моделирования при разных внешних воздействиях или, например, для моделирования падения на луне (изменив силу тяжести).
\\\\
Для модели с идеальными связами необходимо использовать более точные методы
\subsection{Что можно добавить/изменить}
Как было сказано, можно изменить метод приближенного вычисления на более точный.
\\\\
Доконца реализовать различное управление (например принятие заданного положения и его поддержание с помощью регулируемых сил)