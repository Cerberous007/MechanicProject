\section{Постановка задачи}
   Входные данные представляют собой множество точек на плоскости и матрицу связей.Точки представляют собо шарниры, которые соединяют ребра фигуры. Матрица связей представляет собой матрицу смежности, которая отражает, какие из точек соеденинены ребрами. По этим данным составляется некоторая геометрическая фигура.  Для некоторого упрощения будем считать, что наша фигура представляет собой остовное дерево (т.е. в нашей фигуре нет циклов).
   \\\\Необходимо смоделировать процесс падения нашей фигуры по законам механики.